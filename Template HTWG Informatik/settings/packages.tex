% this file includes the packages that are used

\usepackage[colorinlistoftodos]{todonotes}
% begin todonotes package configuration
% documentation of todonotes package http://ctan.127001.ovh/macros/latex/contrib/todonotes/todonotes.pdf

% write todos inline instead of margin
%\presetkeys{todonotes}{inline}{}

% define todo commands for multiple purposes with different colors
% \todo = general todo
\newcommand{\todogen}[1]{\todo[inline]{#1}} %make general todo inline
\newcommand{\todoref}[1]{\todo[inline, color=yellow]{#1}} % check or add reference
\newcommand{\todoresearch}[1]{\todo[inline, color=violet]{#1}} % more research necessary
\newcommand{\tododetail}[1]{\todo[inline, color=cyan]{#1}} % more details please
\newcommand{\todolayout}[1]{\todo[inline, color=magenta]{#1}} % layout issue to be fixed
\newcommand{\todocheck}[1]{\todo[inline, color=teal]{#1}} % double check or verifiy
\newcommand{\todocorrect}[1]{\todo[inline, color=red]{#1}} % wrong statement, please correct
\newcommand{\todoreview}[1]{\todo[inline, color=blue]{#1}} % this section is ready for review

%end todonotes package configuration

% Only necessary when using official HTWG authorship declaration of authorship as PDF
%\usepackage{pdfpages}

\usepackage{listings} %to use source code
%begin listings configuration
\lstset{language=Python} %Python as default language 

%Think about changing name tag for listings...
%\renewcommand{\lstlistingname}{Algorithm}% Listing -> Algorithm
%\renewcommand{\lstlistlistingname}{List of \lstlistingname s}% List of Listings -> List of Algorithms
%ftp://ftp.fu-berlin.de/tex/CTAN/macros/latex/contrib/listings/listings.pdf
%https://en.wikibooks.org/wiki/LaTeX/Source_Code_Listings
%https://www.macwrench.de/wiki/Kurztipp_-_Quellcodelistings_in_LaTeX
%end listings configuration

%acronyms
\usepackage{acronym}

%add bibliography to toc
%\usepackage{tocbibind} -> can be defined in documentclass
%optional: change name Bibliography -> References https://tex.stackexchange.com/questions/12597/renaming-the-bibliography-page-using-bibtex
\renewcommand{\bibname}{References} 

%graphics
\usepackage{graphicx}



%last package is hyperref -> not necessary because bookmark includes hyperref!
%load hyperref package last to avoid conflicts, because it overrides some commands
%http://latex.hpfsc.de/content/latex_tipps_und_tricks/links_urls_pdf/
%\usepackage{hyperref}

%use hyperref instead of bookmark
%check hyperref package for linking
%bookmarks in PDF file
\usepackage[numbered]{bookmark}
%ftp://ftp.mpi-sb.mpg.de/pub/tex/mirror/ftp.dante.de/pub/tex/macros/latex/contrib/oberdiek/bookmark.pdf


